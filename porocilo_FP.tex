\documentclass[12pt,a4paper]{amsart}
% ukazi za delo s slovenscino -- izberi kodiranje, ki ti ustreza
\usepackage[slovene]{babel}
%\usepackage[cp1250]{inputenc}
\usepackage[T1]{fontenc}
\usepackage[utf8]{inputenc}
\usepackage{amsmath,amssymb,amsfonts}
\usepackage{url}
\usepackage{graphicx}
%\usepackage[demo]{graphicx}
%\usepackage[normalem]{ulem}
\usepackage[dvipsnames,usenames]{color}
\usepackage{hyperref}
\hypersetup{
     colorlinks   = true,
     citecolor    = gray
}

% ne spreminjaj podatkov, ki vplivajo na obliko strani
\textwidth 15cm
\textheight 24cm
\oddsidemargin.5cm
\evensidemargin.5cm
\topmargin-5mm
\addtolength{\footskip}{10pt}
\pagestyle{plain}
\overfullrule=15pt % oznaci predlogo vrstico


% ukazi za matematicna okolja
\theoremstyle{definition} % tekst napisan pokoncno
\newtheorem{definicija}{Definicija}[section]
\newtheorem{primer}[definicija]{Primer}
\newtheorem{opomba}[definicija]{Opomba}

\renewcommand\endprimer{\hfill$\diamondsuit$}


\theoremstyle{plain} % tekst napisan posevno
\newtheorem{lema}[definicija]{Lema}
\newtheorem{izrek}[definicija]{Izrek}
\newtheorem{trditev}[definicija]{Trditev}
\newtheorem{posledica}[definicija]{Posledica}


% za stevilske mnozice uporabi naslednje simbole
\newcommand{\R}{\mathbb R}
\newcommand{\N}{\mathbb N}
\newcommand{\Z}{\mathbb Z}
\newcommand{\C}{\mathbb C}
\newcommand{\Q}{\mathbb Q}

% ukaz za slovarsko geslo
\newlength{\odstavek}
\setlength{\odstavek}{\parindent}
\newcommand{\geslo}[2]{\noindent\textbf{#1}\hspace*{3mm}\hangindent=\parindent\hangafter=1 #2}

% naslednje ukaze ustrezno popravi
\newcommand{\program}{Finančna matematika} % ime studijskega programa: Matematika/Finan"cna matematika
\newcommand{\imeavtorja}{Tina Bertok\\ Neža Habjan \\ Gašper Letnar} % ime avtorja
\newcommand{\imementorja}{prof. dr. Riste Škrekovski} % akademski naziv in ime mentorja
\newcommand{\naslovdela}{Genetski algoritem na problemu potujočega trgovca}
\newcommand{\letnica}{2018} %letnica 




\begin{document}

% od tod do povzetka ne spreminjaj nicesar
\thispagestyle{empty}
\noindent{\large
UNIVERZA V LJUBLJANI\\[1mm]
FAKULTETA ZA MATEMATIKO IN FIZIKO\\[5mm]
\program\ -- 1.~stopnja}
\vfill

\begin{center}{\large
\imeavtorja\\[2mm]
{\bf \naslovdela}\\[10mm]
Projekt v povezavi z OR\\[1cm]}

\end{center}
\vfill

\noindent{\large
Ljubljana, \letnica}
\pagebreak

\thispagestyle{empty}
\hypersetup{linkcolor = black}
\tableofcontents
\pagebreak

\section{Navodilo}

Implement the genetic algorithm metaheuristic for TSP. Present the chromosomes as ordered
lists i.e. as paths. Apply different variations for the crossover operations, such as order crossover
(OX), paritally mapped crossover (PMX), cycle crossover (CX), etc. Experiment with different
sizes of the population. You can generate some of the testing graphs yourself, and you can find
some of them on the Internet.

\newpage
\section{Uvod}

\newpage
\section{Opis dela}

\

\newpage
\section{Zaključek}


\begin{thebibliography}{99}


\bibitem{wiki}
\emph{Genetic algorithm}, v: Wikipedia: The Free Encyclopedia, [ogled 13.~12.~201], dostopno na \url{https://en.wikipedia.org/wiki/Genetic_algorithm}.

\bibitem{splet}
N.~Kumar, Karambir in R.~Kumar, \emph{A Comarative Analysis of PMX, CX and OX Crossover operators for solving Travelling Salesman Problem}, [ogled 13.~12.~2018], dostopno na \url{http://www.mnkjournals.com/ijlrst_files/Download/Vol%201%20Issue%202/303-%20Naveen.pdf}.

\end{thebibliography}






\end{document}